% Chapter Template


\chapter{Introduction} % Main chapter title

\label{introduction} % Change X to a consecutive number; for referencing this chapter elsewhere, use \ref{ChapterX}
\begin{flushleft}
Human 2D pose estimation has gained more and more attraction in recent years.
For example Facebook, one of the BigFive technology companies, has published 73 research paper targeting the problem of pose estimation in the last three years.
The most popular ones are DensePose and VideoPose3d~\cite{fbPub, DensePose, videopose3d}.
Furthermore, many enterprises are becoming more and more interested in Sport Content Analysis (SPA)
e.g. Bloomberg, SAP and Panasonic, just naming a few~\cite{sappanasonic, spaBloomberg}.
\end{flushleft}
\begin{flushleft}
But how got this topic into such a demanded focal point?
Probably this is due to the various application areas in which Pose estimation can be encountered.
Main fields are sports, visual surveillance, autonomous driving, entertainment, health care and robotics \cite{olympicsport, surveillance, kinectWalkDepression}.
For example Vaak, a japanese startup, developed a software, which would detect shoplifters,
even before they were able to remove items from a store.
This yielded in a drastic reduction of stealing crimes in stores.
\end{flushleft}
\begin{flushleft}
The exercise of Sport not via visiting a sports course, gym or club became of fundamental severance in 2020,
when the Coronavirus SARS-CoV-2 spread the entire world~\cite{coronarki}.
Many courses such as Yoga, Pilates or general fitness routines went online
and were often conducted via Zoom, Instagram live or other video streaming technologies~\cite{coronalife}.
However, what participants were often missing, was the feedback of the coach on how the exercise was going, and whether it was done right or wrong.
So in 2020 more than ever was missed a technology which is good at pose estimation, or even further, action recognition, in sports.
\end{flushleft}
\begin{flushleft}
2D Pose estimation sets the baseline for machines to understand actions.
It is the problem of localizing human joints or keypoints in images and videos.
Many research studies explored and researched this topic already with the most popular ones being OpenPose and VideoPose3d~\cite{openpose, videopose3d}.
A popular company in Canada \textit{wrnch.ai} even specialized on keypoint recognition from image and video data with a lot of product options~\cite{wrnch}.
\end{flushleft}
\begin{flushleft}
For 2D pose recognition there are mainly two general procedures: either top-down or bottom-up.
Top-down first detects a person and then finds their keypoints.
Whereas bottom-up first detects all keypoints in the image and then refers the corresponding people.
For top-down it is argued, that if a person is not detected via a bounding box or alike, no keypoints can be found.
This would lead to more unlabeled frames in a video.
When there are many people in the image with many occlusions the people often can not be detected.
However, when a person is correctly detected, it is said that accuracy would be higher~\cite{synergetic}.
OpenPose, as the famous bottom-up approach, shows results where they were able to detect multiple people with their poses in videos.
With the according hardware this would even show decent results in realtime~\cite{openpose}.
\end{flushleft}
\begin{flushleft}
Most investigations in this field target usual activities not including complex poses which can be encountered in professional sport.
This is why these architectures often fail when applied to more complex movements.
For competitive sports there are various metrics of high interest depending on the environment.
Competition and training can be differentiated as can be sports executed by multiple athletes versus single combats.
Basketball or soccer as team sports for example are interested on predictions about
how the other team behaves during the game and which would be the best reaction to their behaviour for winning the game.
During practice 2d pose recognition can help to optimize the sports-person movements by taking the role of a coach.
This could provide an answer to the question on how certain activities might be optimized? \
Single competitive sports with very complex movement routines are for example gymnastics and figure ice skating.
Both sports include various artistic body movements, which are not part of daily activities.
Even famous and well rated 2d pose recognition networks such as OpenPose or VideoPose3d fail to recognize these poses.
\end{flushleft}
\begin{flushleft}
If this problem was solved it could help with action recognition and support during practice or relieve the jury on competitions.
A predictor could for example suggest, what an athlete should do to land a certain jump.
On the other hand jury is rare and the job sitting all day in the ice-rink on weekends with only a very small salary is not very attractive.
Furthermore, people often complain scoring is not executed fairly.
\end{flushleft}
\begin{flushleft}
Especially in figure ice skating an accurate 2d pose recognition could make a huge contribution.
This is why this paper investigates 2d pose recognition with special focus on figure ice skating.



%----------------------------------------------------------------------------------------
%	SECTION 1
%----------------------------------------------------------------------------------------


%More and more online trainings - corona virus: often via Instagram life or Zoom.
%We as humans can tell whether the person does the movement right or not, so it should be possible for neural networks as well to recognize.
%Everyone has a mobile phone with a camera. No difficult setup needed and no additional hardware.
%
%In some real-world scenarios (e.g. vehicle accidents and criminal activi-
%ties), intelligent machines do not have the luxury of waiting for
%the entire action execution before having to react to the action
%contained in it. For example, being able to predict a dangerous
%driving situation before it occurs; opposed to recognizing it
%thereafte
%
%- profi knowns what action is performed
%    - slow motion: can detect possible errors
%    - put videos next to each other: who jumps higher, turns faster - how fast is difficult
%    - errors difficult to spot -> many ways to land a jump
%- general tasks
%- application areas:
%    - police, games, sports, health
%- in sports
%- figure skating as speciality


%----------------------------------------------------------------------------------------
%	SECTION 1
%----------------------------------------------------------------------------------------


\section{Motivation and Goals}

\begin{flushleft}
A working 2d pose estimator could make a huge contribution to figure ice skating. 
Especially when building an action recognizer on top of it. 
However, as of today, this was not possible yet due to the complex poses and the different gliding movements on the ice.
Especially spins with their fast rotation and stretching poses are of high complexity to these estimators.
Such an estimator could support fair scoring during competitions or help to improve motion sequences during practice.
\end{flushleft}
\begin{flushleft}
With the downward trend of jury staff and the increasing demand for more small competitions, jury is asked more and more in figure ice skating.
Particularly the role of the technical specialist or controller diagnosing the individual elements on the ice is of high demand.
Some competitions were event canceled in recent years, because they were not able to find the according jury.
Furthermore, sitting all day in the cold ice rink for only a very low salary in not attractive at all.
These long demanding days challenge concentration and many competition participants often complain about jury not rating fairly enough,
completely forgetting the demanding work the jury has to do.
Here a 2d pose estimator could contribute by recognizing the different elements or even scoring. 
This would not only relieve the jury but also could increase fairness.
\end{flushleft}
\begin{flushleft}
During practice 2d pose estimators could examine the specific motions during elements and give hints how to improve these.
Probably they could even suggest certain exercises to learn an element like a spin, jump or certain step. 
Additionally they could keep track of training and provide analysis metrics to the skaters and coaches.
\end{flushleft}
\begin{flushleft}
All in all 2d pose estimation is very interesting not only because of all the possible different appliance possibilities in this sport,
 but as well because of the
challenging task to build an according estimator, which was not possible until today.
\end{flushleft}
% - einsatzbereiche - trainer-app, preisrichter-knappheit
% - unger memo
% - große komplexität von eiskunstlauf, fachwissen notwendig
% - keine erfolgreichen studies bis jetzt, neulich veröffentlichtes dataset
% Lorem ipsum dolor sit amet, consectetur adipiscing elit. Aliquam ultricies lacinia euismod. Nam tempus risus in dolor
% rhoncus in interdum enim tincidunt. Donec vel nunc neque. In condimentum ullamcorper quam non consequat. Fusce
% sagittis tempor feugiat. Fusce magna erat, molestie eu convallis ut, tempus sed arcu. Quisque molestie, ante a
% tincidunt ullamcorper, sapien enim dignissim lacus, in semper nibh erat lobortis purus. Integer dapibus ligula ac
% risus convallis pellentesque.


\section{Related Work}

Lorem ipsum dolor sit amet, consectetur adipiscing elit. Aliquam ultricies lacinia euismod. Nam tempus risus in dolor
rhoncus in interdum enim tincidunt. Donec vel nunc neque. In condimentum ullamcorper quam non consequat. Fusce
sagittis tempor feugiat. Fusce magna erat, molestie eu convallis ut, tempus sed arcu. Quisque molestie, ante a
tincidunt ullamcorper, sapien enim dignissim lacus, in semper nibh erat lobortis purus. Integer dapibus ligula ac
risus convallis pellentesque.
