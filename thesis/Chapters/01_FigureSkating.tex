% Chapter Template


\chapter{Figure Skating Pose Detection} % Main chapter title

\label{figureskating} % Change X to a consecutive number; for referencing this chapter elsewhere, use \ref{ChapterX}

%----------------------------------------------------------------------------------------
%	SECTION 1
%----------------------------------------------------------------------------------------


\section{Complexity of Figures}

Figure ice skating includes very special movement sequences that stand out from other sports.
Thanks to the surface of the ice, skaters perform gliding movements.
Furthermore, their programs include highly artisitic
movements with explosive take-offs, very high rotation speeds, and flexible positions.
Current pose estimators such as \textit{OpenPose, VideoPose3d} or \textit{wrnch.ai} all fail to correctly predict keypoints
for spins with their high
rotation speed and difficult flexible positions.
In this section, we try to explain why especially figure ice skating means a challenge in human pose estimation.
\\\mbox{}\\
In single figure ice skating, there are three main element types in a competitive program, which receive
technical scores:
jumps, steps, and spins.
There are seven different listed jumps, which only differ in their take-off phase.
Three
of these
are jumped just from one edge, the Axel, Loop, and Salchow.
The other ones additionally use the skate toe as a catapult.
In fact, they can be distinguished by the edge that is last skated on the ice before the skater takes off.
This is one of the reasons, why many skaters have problems with the Lutz and Flip jumps.
For the jury it is often hard to tell as well, whether the jump should get an edge deduction.
All the jumps can include a different amount of rotation in the air,
whereas four rotations were the maximum a skater could perform until today.
All these jumps can be combined in various manners, include features such as lifted arms or difficult steps
before or after the element to
increase the level of difficulty, resulting in higher scores.
\\\mbox{}\\
For spins, there are four main positions: upright, sit, camel and layback.
These positions as well can be combined with various features, as for example jumps or difficult elastic positions
such as the famous Biellmann position.
\\\mbox{}\\
There are plenty of different step sequence elements including turns on the ice and steps.
These as well can be combined with multiple additional features resulting in higher scores.
\\\mbox{}\\
The above-named elements only explain the absolute basics in figure ice skating.
In practice scoring and the
creation of ice skating
programs is much more difficult.
Nevertheless, this illustrates nicely, that however for the not professionalized
audience many elements may
look the same, there are plenty of different metrics and components.
Moreover, each skater has their own style
and performs these elements
slightly different.
This is what makes it so hard for machine-learning algorithms to correctly predict the elements.
Which is
why it is important that
the basis, the keypoint recognition module, has to predict the poses correctly.
Because otherwise the action
recognition module has no
chance to make correct estimations~\cite{isuguideleinesgoe, isujudginssystem}.

%- existing KP detectors struggle (OpenPose, VideoPose)
%
%Lorem ipsum dolor sit amet, consectetur adipiscing elit. Aliquam ultricies lacinia euismod. Nam tempus risus in dolor
%rhoncus in interdum enim tincidunt. Donec vel nunc neque. In condimentum ullamcorper quam non consequat. Fusce
%sagittis tempor feugiat. Fusce magna erat, molestie eu convallis ut, tempus sed arcu. Quisque molestie, ante a
%tincidunt ullamcorper, sapien enim dignissim lacus, in semper nibh erat lobortis purus. Integer dapibus ligula ac
%risus convallis pellentesque.

%----------------------------------------------------------------------------------------
%	SECTION 2
%----------------------------------------------------------------------------------------


\section{Distinct Rating System}
First reported figure ice skating competitions with a registered scoring system took place in Vienna in 1881.
The International Skating Union (ISU) regulates figure ice skating competitions since 1892 and is responsible for the
construction
and further development of its judging system.
The old, still very well known judging system \textit{6.0} was officially used until the world championships in 2004,
and
is still used very rarely at tiny competitions or fun competitions, due to no special technical equipment requirements.
This system allows grades from 0-6 with 6 standing for best outstanding skating.
Further refinements could be rated via decimal numbers.
A program receives grades \textit{A} and \textit{B}, with the A-Grade rating technical skills and the B-Grade judging
performance gratitude and expression.
Both grades added resulted in the placement of the skater, where placement proposals of the judges where combined via
the majority principle, resulting in the final placement.
\\\mbox{}\\
Since the more and more rising complains~\cite{unfairjudge}, a new system was developed to increase fairness, today
referenced as ISU
Judging System.
This new system was adjusted several times since it's first introduction in 2004, and only recently in 2019 got updated
fundamentally in it's GOE scores.
It assigns a base value to every jump, step sequence, or spin type in the still existing technical \textit{A-Grade}
part.
Furthermore, several features rise the GOE for jumps, as for example lifted arms or the base value as for spins.
The step sequences as well include special regulations to achieve certain levels and gain a higher base score.
These base scores can become higher or get deductions via the GOE part.
For example, a fallen jump gets its jump value with a GOE of -5.\\
The \textit{B-Grade} judges the skating skills, program transitions, performance or execution, choreography and
interpretation.
Also costume choice and interpretation of music play part in this grade.
Great about the new system is, that the programs are recorded via video, and the technical specialists can review,
slow down
and zoom into the element.
The GOEs then are given by separate judges.
\\\mbox{}\\
Another benefit of this new system is the comparability between competitions and the transparent display of how a
certain final
score was achieved.
Skaters can even collect points from several competitions to get special sponsorships as for example the Kader.
\\\mbox{}\\
One drawback however is the complexity of the new judging system.
Moreover, it requires special technical equipment calculating all the points and for reviewing the programs.
Additionally today, all programs have very specific rules, which elements are allowed or have to be performed.
Which is why some people complain about the resemblance of the different programs and the missing own interpretation
freedom.
\\\mbox{}\\
However, the traceability of scores rose, there is still a very high degree of subjective input in the results,
about which many people complain.
This starts from the first stage of the technical specialist rating a jump with certain deductions or not, and then
of course
is passed to the judges, who pass in their GOE estimations for the elements.
The \textit{B-Grade} still remains very subjective and unfairness is a topic at almost every
competition~\cite{unfairjudge, isujudginssystem, isuguideleinesgoe}.
