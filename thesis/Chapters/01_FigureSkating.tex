% Chapter Template


\chapter{Figure Skating Pose Detection} % Main chapter title

\label{figureskating} % Change X to a consecutive number; for referencing this chapter elsewhere, use \ref{ChapterX}

%----------------------------------------------------------------------------------------
%	SECTION 1
%----------------------------------------------------------------------------------------


\section{Complexity of Figures}

    Figure ice skating includes very special movement sequences which stand out from other sports.
    Thanks to the surface of the ice, skaters perform gliding movements.
    Furthermore, their programs include highly artisitic
    movements with explosive take-offs, very high rotation speeds and flexible positions.
    Current pose estimators such as OpenPose, VideoPose3d or wrnch.ai all fail to correctly predict
    spins with their high
    rotation speed and difficult flexible positions.
    In this section we try to explain why especially figure ice skating means a challenge in human pose estimation.
    \\\mbox{}\\
    In single figure ice skating, there are three main element types in a competitive program, which receive
    technical scores:
    jumps, steps, and spins.
    There are seven different listed jumps, which only differ in their take-off phase.
    Three
    of these
    are jumped just from one edge, the Axel, Loop and Salchow.
    The other ones additionally use the skate toe as a catapult.
    Furthermore, they can be distinguished by the edge that is last skated on the ice before the skater takes off.
    This is one of the reasons, why many skaters have problems with the Lutz and Flip jumps.
    For the jury it is often hard to tell as well, weather the jump should get an edge deduction.
    All the jumps can include a different amount of rotation in the air,
    whereas four rotations was the maximum a skater could perform until today.
    All these jumps can be combined in various manners, include features such as lifted arms or difficult steps
    before or after the element to
    increase the level of difficulty, resulting in higher scores.
    \\\mbox{}\\
    For spins there are four main positions: upright, sit, camel and layback.
    These positions as well can be combined with various features, as for example jumps or difficult elastic positions
    such as the famous Biellmann spin.
    \\\mbox{}\\
    There are plenty of different step sequence elements including turns on the ice and steps.
    These as well can be combined with multiple additional features resulting in higher scores.
    \\\mbox{}\\
    The above named elements only explain the absolute basics in figure ice skating.
    In practice scoring and the
    creation of ice skating
    programs is much more difficult.
    Nevertheless, this illustrates nicely, that however for the not professionalized
    audience many elements
    look the same, there are plenty of different metrics and elements.
    Moreover, each skater has their own style
    and performs these elements
    slightly different.
    This is what makes it so hard in machine-learning to correctly predict the elements.
    Which is
    why it is important that
    the basis, the keypoint recognition module, has to predict the poses correctly.
    Because otherwise the action
    recognition module has no
    chance to make correct estimations.

%- existing KP detectors struggle (OpenPose, VideoPose)
%
%Lorem ipsum dolor sit amet, consectetur adipiscing elit. Aliquam ultricies lacinia euismod. Nam tempus risus in dolor
%rhoncus in interdum enim tincidunt. Donec vel nunc neque. In condimentum ullamcorper quam non consequat. Fusce
%sagittis tempor feugiat. Fusce magna erat, molestie eu convallis ut, tempus sed arcu. Quisque molestie, ante a
%tincidunt ullamcorper, sapien enim dignissim lacus, in semper nibh erat lobortis purus. Integer dapibus ligula ac
%risus convallis pellentesque.

%----------------------------------------------------------------------------------------
%	SECTION 2
%----------------------------------------------------------------------------------------


\section{Distinct Rating System}
- isu scores~\cite{isuscores}
- isu guidelines/ level of execution goe~\cite{isuguideleinesgoe}
- human struggle as well -> rating system with points, many abstractions, still often experienced as not fair

Sed ullamcorper quam eu nisl interdum at interdum enim egestas. Aliquam placerat justo sed lectus lobortis ut porta
nisl porttitor. Vestibulum mi dolor, lacinia molestie gravida at, tempus vitae ligula. Donec eget quam sapien, in
viverra eros. Donec pellentesque justo a massa fringilla non vestibulum metus vestibulum. Vestibulum in orci quis
felis tempor lacinia. Vivamus ornare ultrices facilisis. Ut hendrerit volutpat vulputate. Morbi condimentum venenatis
augue, id porta ipsum vulputate in. Curabitur luctus tempus justo. Vestibulum risus lectus, adipiscing nec
condimentum quis, condimentum nec nisl. Aliquam dictum sagittis velit sed iaculis. Morbi tristique augue sit amet
nulla pulvinar id facilisis ligula mollis. Nam elit libero, tincidunt ut aliquam at, molestie in quam. Aenean rhoncus
vehicula hendrerit.
