%! Author = nadin-katrin
%! Date = 03.04.20
\chapter{Conclusion and future thoughts}
\label{conclusion}
Inspired by popular research from \textit{OpenPose, VidePose3d} or \textit{Simulation of figure skating} ~\cite{openpose, videopose3d, figureskatingsimulation}
which all have problems to correctly predict human joint locations for spins in figure ice skating, this thesis analyzed
several aspects such as motion capture with connection to dataset creation techniques and network architectures with influence of optimizer and loss functions.
\\\mbox{}\\
The result of this research work is a successfully created new high resolution fully convolutional neural network HRNetv3, which includes state-of-the-art
research findings from I.a. HRNet, HRNetV2, and MobileNet~\cite{HRNetv1, HRNetv2, mobilenet}.
This algorithm accomplished to predict human joint locations by learning from the synthetic dataset 3DPeople~\cite{3dpeople}.
In sum, three modules for background extraction, human parts detection, and keypoint recognition were created.
These modules partly correspond to concepts used in other state-of-the-art research such as \textit{OpenPose}, however, the
single ice skating domain specific background extraction module stands out other architectures.
\\\mbox{}\\
For the applied research, a Python 3.7 project  was built with Tensorflow 2~\cite{tensorflow2} as core library.
Several experiments were performed in multiple Docker containers with the Tensorflow:\textit{latest-devel-gpu} as base
image~\cite{tensorflowdocker}
In fact, lots of efforts was spent to write good readable code with a decent OO-style following the \textit{SOLID} principle.
So our Github repository, which is planned to be made publicly available, might help to promote further research in this topic.
Especially, the in this work used feature-driven approach making use of Github's feature pull request style and the formulation of proper
commit messages helped to iteratively improve our project architecture in a continuous style.
\\\mbox{}\\
Since the data is one of the main aspects deciding whether a neural network learns the desired predictions, a look at
motion capturing was taken and the generation of an according dataset investigated.
Further, an appropriate figure ice skating dataset for human joint recognition did not exist at the writing time of this thesis.
Some recordings were captured with the inertial motion capturing set from XSens.
The setup felt very time-consuming and the cables all over the body connected to the battery and sensors felt very uncomfortable
for the skaters on the ice.
Additionally, multiple calibrations had to be conducted, probably because the system was not prepared
for the typical gliding movements on the ice surface.
Nevertheless, after the imposing capture, the integration into Blender and use of MakeHuman for creating a synthetic dataset
became very promising.
Especially the diversity of data or video sequences possible from just one motion capture recording is highly valuable.
This is why it is very convincing that this is the way to gain decent data for an artistic sport such as figure
ice skating.
Yet, critically reviewing \textit{3DPeople}, where random background images were put behind the moving actors,
results could improve, if an ice arena simulation was added to Blender as a more realistic scene.
Another encountered caveat regarding spins was the huge amount of motion blurriness especially for skaters with high levels such as olympic or world championships winning
Evgenia Medvedeva or Alina Zagitova~\cite{2018world}.
Nevertheless, the data from \textit{3DPeople} included only image sequences without any blurriness, being one of the reasons why
the algorithm has difficulties to correctly predict human joints for spins such as the Biellmann pirouette.
\\\mbox{}\\
Throughout this research, further ideas on how to continue in this topic of human joint recognition arose.
First, the creation of a decent synthetic dataset for figure ice skating with added motion blurriness and a more realistic
background scene for the recordings in Blender.\\
Furthermore, test other motion capturing methods as for example the \textit{Awinda} set from XSens, which could be much more comfortable
and faster to apply, because the sensors are wireless, and no cables restricting movements on the ice, must be dealt with.
Additionally, tests could be conducted with a markerless system such as Vicon~\cite{mocapoptical}.
\\\mbox{}\\
Moreover the network architecture could become more compact and better in performance by applying a grid search with multiple different
parameters.
In fact, the temporal information from a video could add additional information, allowing to faster and more efficiently
predict videos as already argued in ~\cite{staf}.
\\\mbox{}\\
This here presented study tries to further spur development and research in figure ice skating to improve
fairness in this sport.
For example, in\cite{figureskatingsimulation}, they simulated some figure ice skating figures and translated these to 3d
animation. It would be very interesting to see, how skaters were rated, if the technical specialists and judges would see
the animation, without knowing, who the skater was, or what the skater looked like.
This could be a huge step in fairness.
In addition, the rating could be conducted remotely, saving the jury from the cold
ice rink and driving efforts.\\
An even more advanced step would be to replace the jury partly, who are sometimes hard to find for a competition.
Another application would be the support during practice, with an included action recommendation compontent could help for instance to improve some jumps.
\\\mbox{}\\
All in all, the here presented research is meant to serve as a foundation for further investigations towards action recognition
in sports, especially artistic ones such as figure ice skating.
Indeed, a very up-to-date topic on the writing of this paper are restrictions during the lock-downs of cities due to the
COVID-19 (Corona) virus.
Highly promising topics include as well physiotherapy or feedback during fitness and dance routines.
