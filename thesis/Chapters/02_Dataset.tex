% Chapter Template


\chapter{Dataset} % Main chapter title

\label{dataset} % Change X to a consecutive number; for referencing this chapter elsewhere, use \ref{ChapterX}

%----------------------------------------------------------------------------------------
%	SECTION 1
%----------------------------------------------------------------------------------------


\section{Figure Skating Dataset}

The correct dataset with the according labels, as the key factor of a neural network architecture, decides whether keypoints
will be learned successfully from image- or video sequence data.
As of today, there does not exist a publicly available dataset which labels keypoints to ice skating performances or
elements.
Famous existing ones such as FIS-V or the MIT datasets only include the total technical and performance scores of the
complete performances, which are not applicable here as explained in \autoref{figureskating}.
\\\mbox{}\\
To create a reasonable dataset for figure ice skating there are various possibilities.
Most straightforward would be to make use of the huge amount of videos published everyday on online platforms such as
\textit{YouTube} and label the videos by hand. This however would cost a huge amount of time and includes outliers coming from
human errors.
Another possibility would be to work with motion capture.
\\\mbox{}\\
%# Neuron: what is motion capture
%- https://neuronmocap.com/content/mocap-101-what-motion-capture


\subsection{Motion capturing techniques}
\label{motioncapture}
There are three main methods for recording motion capturing data.\\
Markerless capture is a method where videos are recorded via a single or multiple depth cameras.
This comes with the advantage that the athletes are not distracted or feel uncomfortable by any
markers~\cite{mocapopticalradical, mocapopticalcapture, mocapmarkerless3dscans}.\\
When working with markers there are two sensor types: optical or inertial.
Optical markers are reflective markers, which are attached to characteristic parts of the body.
These reflections are tracked via multiple cameras to make sure the motion movements are taken especially when a
person is moving, so the markers are not concealed~\cite{mocapoptical}.\\
Inertial motion capture uses IMU sensors that are attached to body parts~\cite{xsens, mocapinterialneuron}.
3D positions then are calculated via multiple sensor metrics.
The XSens system for example includes several trackers that contain magnetometers, gyroscopes, and accelerometers.
Via their MVN Animate and Anlaze systems, the tracked data is post-processed on a biomechanical model
and then can be used in game engines such as \textit{Unity} or \textit{Unreal}, or in 3D animation software such as \textit{Blender} or
\textit{3DSMax}~\cite{xsensabout}.
% https://www.xsens.com/inertial-sensor-modules
% https://www.xsens.com/inertial-sensors
\\\mbox{}\\
In an environment such as the ice rink, motion capture has to deal with several difficulties.
For one, the lighting is often not smooth and there are several different light sources.
Moreover, it is very difficult to get the whole ice rink for a recording, because most of the time the ice rink is fully
booked with figure ice skating, ice hockey, public skating or short track speed skating.
Further, many ice rinks are governed by the local community~\cite{stuttgarteiswelt}, or are highly expensive to rent.
This makes markerless and optical recordings very challenging.
On the other hand, inertial recordings as by the XSens technology seem very promising.
Even when there are multiple skaters on the ice, recordings can be tracked in usual practice time slots.

\subsection{XSens: Inertial motion capturing recordings on the ice rink}
In our research study, we tested recordings on the ice rink with the MVN Link product from XSens.
This product includes 17 wired IMU motion trackers which can be either attached into a suit or onto straps.
The sensors are connected via cables to a battery with a lifetime of 9.5 h.
These trackers are then connected via a router to their MVN Animate software.
The wireless range of the trackers is 50 to 150 meters, however, the trackers are able to buffer the recordings
and transmit these later when there is a connection to the hub.
Their MVN Animate software helps with calibration and shows the recording results as soon as the trackers are close
enough
for transmission.
Furthermore, it processes the data and performs adjustments in a postprocessing step~\cite{xsensmvnanimate}.



\subsubsection*{Our experiences with MVN Link from XSens}
The overall setup was very time-consuming. To correctly attach all the straps and align the cables took us a
minimum of 20 minutes. Here we tried out the suit and the straps with equal preparation times.
Then the calibration on the ice with the MVN Analyze software was very time-consuming again and took about 20 minutes.
The calibration process had to be repeated several times when the calibration failed.
These problems probably came from the different gliding movements on the ice.
So the skaters had to mimic usual floor movements for the calibration to work correctly.
Since the recordings are very time-consuming we could not work with young professional competitive skaters, because
the setup was too drawn-out.
We then took recordings from a senior competitive skater. The inertial method was very beneficial here because other
skaters on the ice, did not intervene in the recordings. Additionally, however, sometimes the connection was lost to
the hub, when the skater came back, the recordings were transmitted correctly.
The post-processing from their software did work really good as well.
What was disruptive, however, were the sensors and battery, which the skater had to carry.
This made the movements feel constrained. Indeed, in figure ice skating falls are very common.
A fall on the battery or one of the sensors would have been harmful, which again influenced the movement of the
skaters.
In comparison, there are other products on the market, that promise a faster setup e.g. through a suit in which the
sensors
are integrated without cables~\cite{mocapinterialneuron}. These products would very likely better fit for figure ice
skating
motion captures.



\subsubsection*{Dataset Creation with Blender and MakeHuman}
We processed the recorded data from the XSens trackers in \textit{MVN Animate} and then exported a short sequence as BVH file
to test possible dataset creation procedures.
Further, we used the open-source \textit{MakeHuman}~\cite{makehuman} program to create a figure ice skating avatar.
Here we used the figure ice skating dress and ice skates, which are freely available from the MakeHuman
community~\cite{makehumanassets}.
With the help of the MakeHuman plugin in Blender 2.7~\cite{blender} we then imported the created avatar from MakeHuman
and retargeted the BVH motion capture file onto the rig.
We then set up a default scene with lighting and a moving camera.
To obtain the keypoints, we wrote a Python script in Blender to calculate the joint locations in relation to the camera
view and exported the resulting keypoints into NumPy files.
\\\mbox{}\\
However, the above-described process seemed to be smooth to conduct, we did face a couple of challenges.
For one, artistic elastic movements yielded in errors on the armature, since the joints were restricted to certain
regions of movement, which seemed to work for usual daily activities but not figure ice skating ones.
Another one was that with the moving camera the keypoints were not calculated correctly and drawn with a small offset.
Recent research studies suggest to export the animation as BVH files and later calculate the joint positions with a
program such as Matlab~\cite{synpose300, 3dpeople, gaitblender}. So this could improve the calculations of joint
positions.



\subsubsection*{Our conclusions on creating a synthetic dataset in figure ice skating}
We are convinced that the above-described process to create a synthetic dataset for figure ice skating is mandatory
when it comes to figure ice skating,
because of the following reasons:\\
First, with inertial motion tracking is very accurate especially in an environment such as an ice rink when recording
can be easily be done during practice sessions.\\
Second, much fewer recordings are necessary thanks to the large arbitrary degrees of variation which is possible with
an animation software such as blender.\\
Third, the diversity of data is exceptionally vast.\\
From one small motion capture file, multiple different avatars can be retargeted, female and male, age variations,
clothes and race variances.
Further, multiple different light sources and camera views can be applied and the background randomly selected.\\
Fourth, calculated labels by the associate animation software or post-processing program are just more exact than
human labeled data.
\\\mbox{}\\
All these points confirm the legitimacy of synthetic datasets in figure ice skating.
Which is why we researched on already existing synthetic datasets since in the scope of this thesis creating our own
dataset was not possible due to the huge time efforts.
The synthetic dataset 3DPeople as described in the next section corresponds to our expectations of an appropriate dataset.


%----------------------------------------------------------------------------------------
%	SECTION 2
%----------------------------------------------------------------------------------------


\section{Synthetic Dataset: 3DPeople}

\begin{figure}
    \centering
    \includegraphics[width=130mm]{Figures/woman_step_3dpeople2.png}
    \decoRule
    \caption[3DPeople: Woman stiff walk (labeled)]{Learned labels from an image sequence of the 3DPeople dataset: Woman stiff walk}
    \label{fig:woman_step_3dpeople}
\end{figure}


The dataset 3DPeople was created via recorded motion capture files.
They took 70 realistic action sequences from Mixamo~\cite{mixamomotionpac}.
The sequences contain on average 110 frames and range from usual activities with little motions such as driving
or drinking to breakdance moves or backflips.
These actions they then retarget onto 80 armatures which are created with MakeHuman~\cite{makehuman} or Adobe
Fuse~\cite{adobefuse}.
There are 40 female and 40 male characters with plenty of variation in body shapes, skin tones, outfits, and hair.
In detail, they record the actions with a projective camera and 800 nm focal length from four different viewpoints
which are orthogonally aligned with the ground.
The distance to the subjects varies arbitrarily as do light sources and static background images.
In sum, the dataset includes 22,400 clips with the rendered video sequences as 640x480 pixel image frames, the associate
depth maps, optical flow, and semantic information such as body parts and cloth labels~\cite{3dpeople}.
\\\mbox{}\\
We found this dataset highly sophisticated due to the wide variance and diversity of the data.
Especially in terms of clothes, it stands out from other famous datasets such as Human3.6M~\cite{humaneva}.
They use a variety of wide and tight clothes. For example do they include dresses and shorts.
There variance is greater compared to their freely available combats~\cite{human36m, humaneva}.
In \autoref{fig:woman_step_3dpeople} we demonstrate the learned labels of our modules from a random image of the
3DPeople dataset.



%----------------------------------------------------------------------------------------
%	SECTION 3
%----------------------------------------------------------------------------------------


\section{Data Processing}
The 3DPeople dataset can be requested through their project homepage and is allowed for use in research
environments.
Each data package consists of five batches for women and men. There are five data packages available for download.
We created a \textit{data\_admin python class}, which takes care of downloading the data, processing and storing the
data
in memory agnostic and efficient access structured compressed NumPy files, and deletes the initial downloaded data.\\
We included several processing steps:
In sum, we created four compressed numpy archives.
Every archived file included an array with all frames of one movement sequence to allow a faster reading process for
our neuronal network training later.
The four archives consist of the usual RGB sequence frames, one archive without background, one for the body part
labels,
and one for the joint keypoints.
The NumPy archive without background only shows the actors in RGB colors.
Therefore, we used the clothes labels and replaced all colored pixels with black pixels when the pixels were not
inside the mask.
Initially, we did this to test, weather our network ideas would converge with easier data.
Later, when we trained our background-extractor module for getting an end-to-end architecture, this data served as
labels.\\
Furthermore, we cleaned up the body masks and mapped the resulting three-dimensional pixels to one-dimensional class
values~\autoref{fig:rgb-mappings}.
The borders of the body-parts often contained a mixture of RGB values, which were of no use to our network.\\
\begin{figure}

    \fontsmall
\begin{minipage}{0.29\linewidth}
    \begin{Verbatim}[frame=topline,label=Mapping to classes,framesep=1mm]
body_part_classes = {
    BodyParts.bg.name: 0,
    BodyParts.Head.name: 1,
    BodyParts.RUpArm.name: 2,
    BodyParts.RForeArm.name: 3,
    BodyParts.RHand.name: 4,
    BodyParts.LUpArm.name: 2,
    BodyParts.LForeArm.name: 3,
    BodyParts.LHand.name: 4,
    BodyParts.torso.name: 5,
    BodyParts.RThigh.name: 6,
    BodyParts.RLowLeg.name: 7,
    BodyParts.RFoot.name: 8,
    BodyParts.LThigh.name: 6,
    BodyParts.LLowLeg.name: 7,
    BodyParts.LFoot.name: 8
}
    \end{Verbatim}
\end{minipage}\hfill
\begin{minipage}{0.49\linewidth}
    \begin{Verbatim}[frame=topline,label=Mapping to RGB values, framesep=1mm]
segmentation_class_colors = {
    BodyParts.bg.name: [153, 153, 153],
    BodyParts.Head.name: [128, 64, 0],
    BodyParts.RUpArm.name: [128, 0, 128],
    BodyParts.RForeArm.name: [128, 128, 255],
    BodyParts.RHand.name: [255, 128, 128],
    BodyParts.LUpArm.name: [0, 0, 255],
    BodyParts.LForeArm.name: [128, 128, 0],
    BodyParts.LHand.name: [0, 128, 0],
    BodyParts.torso.name: [128, 0, 0],
    BodyParts.RThigh.name: [128, 255, 128],
    BodyParts.RLowLeg.name: [255, 255, 128],
    BodyParts.RFoot.name: [255, 0, 255],
    BodyParts.LThigh.name: [0, 0, 128],
    BodyParts.LLowLeg.name: [0, 128, 128],
    BodyParts.LFoot.name: [255, 128, 0]
}
    \end{Verbatim}

\end{minipage}
    \caption{RGB-Classes Mappings}
    \label{fig:rgb-mappings}
    \end{figure}
%\begin{figure}
%    \begin{minipage}{10cm}
%        \label{fig:test}
%    \end{minipage}
%\end{figure}


